\documentclass {article}

\usepackage{amsfonts} % pour les lettres maths creuses \mathbb
\usepackage{amsmath}
\usepackage[T1]{fontenc}
\usepackage[utf8]{inputenc}
\usepackage[frenchb]{babel}
\usepackage{aeguill}
\usepackage{graphicx}
\usepackage{hyperref}
\usepackage{color}

\newcommand\dx{\dot{x}}
\newcommand\dy{\dot{y}}
\newcommand\ddx{\ddot{x}}
\newcommand\ddy{\ddot{y}}
\def\real{{\mathbb R}}

\author {Florent Lamiraux et Steve Tonneau\\
CNRS-LAAS, Toulouse, France}
\title {Tutoriel HPP~: solutions}
\date {}

\begin {document}
\maketitle

\section {Calcul des coefficients de la trajectoire polynomiale de la sortie plate}

On note $P(t) = \sum_{i=0}^{d} t^i P_i$ le polyn\^ome $P$ où
\begin {itemize}
\item l'entier $d$ est le degré du polyn\^ome,
\item les coefficients $P_i$ sont à valeur dans $\real^2$.
\end {itemize}
Etant données deux configurations définies par les vecteurs $(x_0,y_0,c\theta_0,s\theta_0,\zeta_0)$ et $(x_1,y_1,c\theta_1,s\theta_1,\zeta_1)$, pour que la courbe $P$ définisse une trajectoire de la voiture entre ces deux configurations, il suffit que~:
\begin {align}\label{eq:constraints-0}
  P (0) &= (x_0, y_0) \\
  \dot {P} (0) &= (\cos (\theta_0), \sin (\theta_0)) \\
  \ddot {P} (0) &= \kappa_0 (-\sin (\theta_0), \cos (\theta_0)) \\
  P (1) &= (x_1, y_1) \\
  \dot {P} (1) &= (\cos (\theta_1), \sin (\theta_1)) \\
  \label{eq:constraints-1}
  \ddot {P} (1) &= \kappa_1 (-\sin (\theta_1), \cos (\theta_1))
\end {align}
Cela correspond à 12 équations (6 pour chaque coordonnée $x$, $y$) linéaires sur les coefficients de $P$. Rappelons que
$$
\kappa_{0} = \frac {\tan \zeta_{0}}{l} \hspace*{1cm} \kappa_{1} = \frac {\tan \zeta_{1}}{l}
$$
Le degré $d=5$ devrait permettre de trouver une solution.

Exprimons les dérivées de $P$~:
\begin {align*}
P(t) &= P_0 + t P_1 + t^2 P_2 + t^3 P_3 + t^4 P_4 + t^5 P_5 \\
\dot {P}(t) &= P_1 + 2t P_2 + 3t^2 P_3 + 4t^3 P_4 + 5t^4 P_5 \\
\ddot {P}(t) &= 2 P_2 + 6t P_3 + 12t^2 P_4 + 20t^3 P_5 \\
\end {align*}
Ces expressions, combinées avec les contraintes~(\ref{eq:constraints-0}-\ref{eq:constraints-1}), donnent les équations suivantes sur les coefficients de $P$~:
\begin {align*}
P_0 &= (x_0,y_0) \\
P_0 + P_1 + P_2 + P_3 + P_4 + P_5 &= (x_1,y_1) \\
P_1 &= (\cos(\theta_0), \sin(\theta_0)) \\
P_1 + 2 P_2 + 3 P_3 + 4 P_4 + 5 P_5 &= (\cos(\theta_1), \sin(\theta_1))\\
2 P_2 &= \kappa_0 (-\sin (\theta_0), \cos (\theta_0)) \\
2 P_2 + 6 P_3 + 12 P_4 + 20 P_5 &= \kappa_1 (-\sin (\theta_1), \cos (\theta_1))
\end {align*}
Ce système linéaire admet une solution unique~:
\begin {align*}
P_0 &= \left(\begin{array}{c}x_0\\y_0\end{array}\right) \\
P_1 &= \left(\begin{array}{c}\cos(\theta_0)\\ \sin(\theta_0)\end{array}\right) \\
P_2 &= \frac{\kappa_0}{2} \left(\begin{array}{c}-\sin (\theta_0)\\ \cos (\theta_0)\end{array}\right) \\
P_3 &= 10 \left(\begin{array}{c}x_1-x_0\\y_1-y_0\end{array}\right) -6 \left(\begin{array}{c}\cos(\theta_0)\\\sin(\theta_0)\end{array}\right) -4 \left(\begin{array}{c}\cos(\theta_1)\\\sin(\theta_1)\end{array}\right) -\frac{3}{2} \kappa_0\left(\begin{array}{c}-\sin(\theta_0)\\\cos(\theta_0)\end{array}\right) + \frac {1}{2}\kappa_1\left(\begin{array}{c}-\sin(\theta_1)\\\cos(\theta_1)\end{array}\right)\\
P_4 &= -15 \left(\begin{array}{c}x_1-x_0\\y_1-y_0\end{array}\right) +8 \left(\begin{array}{c}\cos(\theta_0)\\\sin(\theta_0)\end{array}\right) +7 \left(\begin{array}{c}\cos(\theta_1)\\\sin(\theta_1)\end{array}\right) +\frac{3}{2} \kappa_0\left(\begin{array}{c}-\sin(\theta_0)\\\cos(\theta_0)\end{array}\right)  - \kappa_1\left(\begin{array}{c}-\sin(\theta_1)\\\cos(\theta_1)\end{array}\right)\\
P_5 &= 6 \left(\begin{array}{c}x_1-x_0\\y_1-y_0\end{array}\right) -3 \left(\begin{array}{c}\cos(\theta_0)\\\sin(\theta_0)\end{array}\right) -3 \left(\begin{array}{c}\cos(\theta_1)\\\sin(\theta_1)\end{array}\right) -\frac{1}{2} \kappa_0\left(\begin{array}{c}-\sin(\theta_0)\\\cos(\theta_0)\end{array}\right) + \frac {1}{2}\kappa_1\left(\begin{array}{c}-\sin(\theta_1)\\\cos(\theta_1)\end{array}\right)
\end {align*}
\end {document}
