\documentclass {article}

\usepackage{amsfonts} % pour les lettres maths creuses \mathbb
\usepackage{amsmath}
\usepackage[T1]{fontenc}
\usepackage[utf8]{inputenc}
\usepackage[frenchb]{babel}
\usepackage{aeguill}
\usepackage{graphicx}
\usepackage{hyperref}
\usepackage{color}

\newcommand\dx{\dot{x}}
\newcommand\dy{\dot{y}}
\newcommand\ddx{\ddot{x}}
\newcommand\ddy{\ddot{y}}
\def\real{{\mathbb R}}

\author {Florent Lamiraux et Steve Tonneau\\
CNRS-LAAS, Toulouse, France}
\title {Travaux pratiques sous HPP}
\date {}

\begin {document}
\maketitle

\section {Introduction}
L'objectif de ce TP est d'acqu\'erir une exp\'erience
de programmation .


virtual box

pres archi

tuto

shoot

clone tp

onestep

rand + extend

extend

adding nodes

connect

biased shoot

pathvalidatio


Une méthode de guidage \texttt{hpp::core::SteeringMethod} est une
classe qui construit des chemins entre des paires de configurations
d'un robot. Son r\^ole est de prendre en compte la cinématique
particulière du robot en question. Par exemple, un robot de type
voiture ne peut pas exécuter n'importe quel mouvement, à cause des
contraintes de roulement sans glissement des roues.

L'objectif de ce tutoriel est de développer une méthode de guidage
respectant les contraintes de roulement sans glissement d'un robot de
type voiture.

Pour cela, les étapes suivantes sont nécessaires~:
\begin {enumerate}
\item dériver la classe \texttt{hpp::core::Path} afin de modéliser des trajectoires respectant les contraintes de roulement sans glissement de la voiture,
\item dériver la classe \texttt{hpp::core::SteeringMethod} afin de créer une méthode de guidage produisant des chemins du type défini à l'étape précédente entre deux configurations données,
\item rendre cette méthode de guidage accessible dans les scripts python.
\end {enumerate}
Le point~1 ci-dessus fait l'objet de la partie suivante.

\subsection {Trajectoires respectant les contraintes de roulement sans glissement}
\subsubsection* {Platitude différentielle}

Considérons la voiture représentée sur la figure~\ref{fig:voiture}. C'est un système dit différentiellement plat, dont la sortie plate est le centre de l'axe des roues arrières $(x,y)$. Cela signifie que la trajectoire dans le plan de $(x,y)$ permet de reconstruire la configuration du système uniquement par dérivation.

En effet, soit
\begin {itemize}
  \item $(x,y)$ la position du centre de l'axe des roues arrières,
  \item $(\dx, \dy)$ sa vitesse,
  \item $(\ddx, \ddy)$ son accélération.
\end {itemize}
A condition que la vitesse soit non-nulle, on peut établir les relations suivantes~:
\begin {align}\label{eq:orientation}
  \cos(\theta) = \frac{\dx}{\sqrt{\dx^2+\dy^2}} & \hspace {1cm} & \sin(\theta) = \frac{\dy}{\sqrt{\dx^2+\dy^2}}
\end {align}
La dérivée première de la sortie plate donne l'orientation du véhicule,
\begin{align}\label{eq:courbure}
  \kappa = \frac{\tan (\zeta)}{l} = \frac{\dx\ddy - \dy\ddx}{(\dx^2+\dy^2)^{\frac{3}{2}}}
\end{align}
La dérivée seconde via la courbure donne l'angle de braquage.

\subsection* {Mise en oeuvre}

Etant données deux configurations définies par les vecteurs $(x_1,y_1,c\theta_1,s\theta_1,\zeta_1)$ et $(x_2,y_2,c\theta_2,s\theta_2,\zeta_2)$, nous proposons de définir une trajectoire faisable entre ces deux configurations par l'intermédiaire d'un polynome défini sur l'intervalle $[0,1]$ à valeurs dans $\real^2$ et décrivant la trajectoire de la sortie plate du système. Nous notons $P$ ce polyn\^ome. Les relations~(\ref{eq:orientation}) et (\ref{eq:courbure}) induisent des équations non-linéaires sur les coefficients du polyn\^ome $P$. Pour rendre ces équations linéaires, il suffit d'ajouter les contraintes suivantes~:
\begin{enumerate}
\item la dérivée première de la sortie plate en 0 et en 1 a pour norme 1,
\item la dérivée seconde de la sortie plate en 0 (resp. 1) est orthogonale à la dérivée première en 0 (resp. 1).
\end{enumerate}

\paragraph {Question 1:} écrire les équations sur les cofficients du polyn\^ome $P$ en fonction des configurations initiales et finales $(x_1,y_1,c\theta_1,s\theta_1,\zeta_1)$ et $(x_2,y_2,c\theta_2,s\theta_2,\zeta_2)$.

\paragraph {Question 2:} quel degré de $P$ permet d'avoir autant d'équations que d'inconnues.

\paragraph {Question 3:} résoudre le système d'équation et exprimer les coefficients de $P$ en fonction des configurations initiales et finales.
\end {document}
